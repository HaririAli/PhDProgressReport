\section{Background and Research Problems}\label{sec:background}
This section describes the state of the art and outline the research problems that need to be addressed.

\subsection{\acrfull{ucon} Model}
The \gls{ucon} model was proposed by Park and Sandhu \cite{ucon} as generalisation of access control that covers Authorisations \((A)\), oBligations \((B)\) and Conditions \((C)\), and grants or denies access based on attributes.
Authorisations are functional predicates over subject and object attributes, and must be satisfied to grant access rights; Obligations are mandatory actions that must be fulfilled in order to grant access rights; and Conditions are environmental or situational requirements that must be met to grant access rights.
The main novelty of \gls{ucon} is that it extends access control with mutability of attributes and continuity of control.
Unlike preceding models, \gls{ucon} assumes that some attribute values may change during usage of resources.
For this reason, it specifies that authorisations, obligations and conditions must be evaluated before access starts, while access is in progress, and after access ends, which adds continuity of control.
Therefore, if an attribute value changes while access is in progress, and the security policy is not satisfied anymore, \gls{ucon} revokes the granted authorisation.
\gls{ucon} classifies the decision predicates of authorisations, obligations and conditions as \(pre\) and \(on\) (i.e., \(preA\), \(onA\), \(preB\), \(onB\), \(preC\) and \(onC\)), where \(pre\) predicates are evaluated before usage starts and \(on\) predicates are evaluated while usage is in progress.
\gls{ucon} also classifies attribute updates as \(preUpdates\), \(onUpdates\) and \(postUpdates\), such that \(preUpdates\) happen before usage starts, \(onUpdates\) take place during usage, and \(postUpdates\) change attribute values after the end of usage.
%TODO: obligations for attribute updates and post obligations for actions after authorisation 
In \cite{park2004mutability}, Park et al. defined a taxonomy for attribute management in \gls{ucon}.
They classified attributes as immutable and mutable, where immutable attributes are the ones that require administrative actions to be changed while mutable attributes are the ones that change as result of usage itself. 
Their taxonomy also defines subcategories under immutable and mutable attributes, but this is out of the scope of this report.
Park et al. assume that attribute updates occur only as a side-effect of usage (i.e., as part of the enforcement of the security policy or due to actions performed by subjects).
They also assume that environmental attributes do not change during usage.
However, as described by Carniani et al. \cite{ucs2016cloudsys} attributes may change due to reasons unrelated to usage (e.g., subject's location changes as they move), and environmental attributes can change during usage (e.g., date, time, CPU load, temperature, etc.).
\gls{ucon} dist




% the model was intended for DRM but it can cover a wide variety of use-cases
% Fabio proposed an architecture and policy language for UCON
% CNR applied UCON in multiple use-cases
% Dimitrakos extended the pre state by adding a counter but this is a hack because counter can expire before the intended objective
% The pre and post states must be extended on the model level
% there is no problem with them having a long lifetime on the model level. This is implementation detail.
% Multiple state models may be needed. 
% The three phases are not enough and there may be more phases.
% Classifying policies by phases is actually unnecessary.
% Decision does not necessarily have an effect on the state and no need to re-evaluate.
% UCON lacks formalism in the sense that it is 


% state of the art
% ucon formalism
% session management in distributed UCON
% context propagation and relationships between authorisations
% continuity of pre and post
% limitation of the three phases
% applications beyond access control that require actions


% \subsection{Stateful Access Control}
% Negotiation before authorisation
% TBD

\subsection{\acrfull{iam}}
% EU Digital Sovereignty 
% GAIA-X 
% SSI 
% Dataspaces and data flow control