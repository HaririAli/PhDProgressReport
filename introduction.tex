\section{Introduction}\label{sec:introduction}
The rise of the ubiquitous Internet (e.g., smartphones, smart wearables, IoT devices, social networks, etc.) enabled the generation, collection and processing of huge amounts of personal data such as location, health records, user behaviour and video feeds \cite{iotdata}.
As a result, data security and user privacy have become major concerns fueled by privacy-related incidents such as the Cambridge Analytica scandal \cite{davies2015ted}.
Privacy laws and regulations, such as the \gls{gdpr} \cite{gdpr}, have been introduced to address these concerns.
Moreover, many initiatives such as GAIA-X and \gls{idsa} \cite{eudigitalsov} have been launched with the objective of controlling data access and usage, enforcing policies, guaranteeing compliance with regulations, and ensuring that digital systems are privacy-preserving.
Such initiatives rely on \gls{iam} solutions, which have become more prevalent as access control and privacy requirements become more complex and resources get distributed across increasingly heterogeneous environments.

%\gls{iam} is the management of the digital representations of real-world entities (e.g., people, organisations, objects, software, etc.) and its association and orchestration with authentication and authorisation solutions that ensure that only authorised entities get access to data and resources \cite{vallois2021blockchain} \cite{chadwick2009federated}.
%Digital identities have evolved from centralised identity to federated identity, and most recently to the emerging \gls{ssi} \cite{sovrin2016inevitable}.
%Centralised identity have been used in the majority of web services such as social networks or eCommerce, where each service has its own \gls{iam} system and its own database of identities.
%Federated identity was introduced to address the accelerated growth of users and services as it allows the use of the same digital identity across different IT systems in heterogeneous environments within the same federation (e.g., enterprise, organisation, etc.).
%It also allows digital identities to be ported between different federations according to established trust relationships \cite{sovrin2016inevitable}\cite{xu2020identity}\cite{shim2005fim}.
%Both federated and centralised identities are fully controlled by organisations rather than the identity owners, which raised privacy concerns.
%For this reason, \gls{ssi} emerged to give identity owners full control over their digital identities, removing the centralised and external control over identities and making identity owners independent of any organisations \cite{sovrin2016inevitable}.
%\gls{ssi} has been gaining a lot of attention and is being adopted in many initiatives such as GAIA-X and \gls{essif}.
%However, there are several problems that are yet to be solved in order to support a wide adoption of \gls{ssi}, especially since \gls{ssi} technologies are still immature, and because organisation have invested heavily in traditional identity management systems such as \gls{fim} or centralised identity.

%Similarly,
Access control models have evolved over the years from \gls{mac} and \gls{dac} \cite{sandhu1993lattice} to the most prominent models, \gls{rbac} \cite{rbac1992nist} and \gls{abac} \cite{abac}.
\gls{rbac} associates permissions with roles that are assigned to subjects and represent job functions, competencies, authorities or responsibilities (e.g., employees are assigned roles such as ``manager'' or ``engineer'' and each role has different access rights) \cite{rbac1992nist}\cite{sandhu1996role}.
\gls{abac} adds more flexibility and granularity by associating access rights with attributes about subjects (e.g., name, address, etc.), objects (e.g., file type, size, etc.) and the environment (e.g., time, location, etc.) \cite{abac}.
\gls{capbac} is also another fine-grained access control model used in many systems, where access rights are assigned to subjects in the form of capabilities issued by an authorisation authority.
Capabilities are typically issued by a \gls{sts} as short-lived tokens that are then presented by the subject to the resource provider which verifies the token and grants or denies access according to the embedded capabilities \cite{fang2005xpola}.
The aforementioned access control models assume that roles/attributes/capabilities, and consequently access rights, do not change during access/usage of resources and as such, they do not react to situation changes.
For this reason, J. Park and R. Sandhu \cite{ucon} proposed the \gls{ucon} model that extends \gls{abac} with mutability of attributes and monitoring of attribute values as well as continuity of access control and policy re-evaluation.
\gls{ucon} gained a lot of attention and many researchers have extended and leveraged it to address different use-cases and applications \cite{ucs2016cloudsys}\cite{ucs2017iot}\cite{ucs2017mqtt}\cite{pretschner2011dataucon}.
We also extended \gls{ucon} with the \gls{adp} in \cite{datapal} in order to introduce a new policy model that supports user consent and intervenability by allowing data owners to define data-centric policies enforced at every step of the data lifecycle (i.e., collection, retention, treatment and deletion).
Although \gls{ucon} extended access control with session-based evaluation and monitoring, and despite the significant works that extended the model and developed it, some limitations still need to be addressed in order to accommodate some emerging use-cases such as data flow control.
In this report, we outline the state of the state of the art of \gls{ucon}, then we discuss the related research problems and our objectives as well as the research progress and plan.

The rest of the report of organised as follows:
Section \ref{sec:background} outlines the state of the art and the research problems.
The corresponding research objectives are described in Section \ref{sec:objectives}.
We discuss the research progress and the future plan to complete the objectives in Section \ref{sec:progress}.
In Section \ref{sec:publications}, we list the progress and plan for publications.
Finally, we define future directions and draw conclusions in Section \ref{sec:conclusion}.