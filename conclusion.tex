\section{Conclusion}\label{sec:conclusion}
\gls{ucon} is an access control model introduced by Park and Sandhu extending previous models with mutability of attributes, continuous monitoring and policy re-evaluations.
It gained a lot of attention in the research community and it has been extended and developed to accommodate different use-cases that focused on \gls{drm} or long-lasting authorisations.
\gls{ucon} divides authorisation sessions into three phases designated as \(pre\), \(ongoing\) and \(post\), such that the \(pre\) phase refers to evaluations that happen before access to resources starts, the \(ongoing\) phase implies that usage has started and continuous monitoring is in place, and the \(post\) phase covers interactions that must happen after the usage is revoked or ended.
In spite of the expressiveness of \gls{ucon} and its support of a wide variety of applications, it still have some restrictions that cannot accommodate some emerging use-cases such as data flow control or trust-aware access control.
For this reason, we are working on extending the \gls{ucon} model to support emerging use-cases and generalising the model into a session-based \gls{abac} model.
In this report, we outlined the research problems and objectives as well as the progress and plan to achieve our goals.
We also introduced a policy-based and context-aware identity and privilege management system with an application in smart vehicles.
we described our plan to use the system as a GAIA-X \gls{iam} middleware that supports the interoperability of identity technologies as well as the delegation and automation of issuing and verifying credentials.
Finally, we proposed two architecture for clustering \acrshort{ucsp} in cloud and enterprise environments and for managing sessions in the clusters.